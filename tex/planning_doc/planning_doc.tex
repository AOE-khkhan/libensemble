\documentclass{article}

\usepackage{listings}
\usepackage{amsmath, amsfonts, amssymb, setspace, color, graphicx}%psfrag}%,pstool}
\usepackage[american]{babel}
\usepackage[normalem]{ulem} % For \sout
\usepackage{fullpage}
\everymath{\displaystyle}
\usepackage[ruled,vlined]{algorithm2e}

\newenvironment{allintypewriter}{\ttfamily}{\par}
\newcommand{\jlnote}[1]{\textsf{{\color{blue}{ JL note:}   #1.} }\marginpar{{\textbf{Comment}}}}
\newcommand{\ps}[1]{\ensuremath{\mathtt{#1}}}
\newcommand{\libE}{\texttt{LibEnsemble}\xspace}

\newcommand{\algref}[1]{{\rm Algorithm}~\ref{alg:#1}}

\title{\libE planning document}

\begin{document}
\maketitle
\begin{abstract}
  This document will be an explicit plan of what will be developed in
  \libE. Proper planning should help ensure that the code will be flexible
  and easy to adjust in the future.

  Current development considers a manager and worker framework. 
\end{abstract}

In order to ensure applicability to a variety of use cases, 
\libE will coordinate many different types of calculations. Examples of
such calculations include:
\begin{itemize}
  \item Simulation evaluations
  \item Generation of point(s) to be evaluated by the simulation(s)
  \item Updating points in the active queue
  \item Updating active simulations
  \item Monitoring intermediate output from simulations
\end{itemize}
Users will only need to provide scripts for simulation evaluation and point
generation. We will provide default scripts for the other types of calculations, 
which can be templates for the interested user. As a default the Manager will
perform many calculations itself (essentially blocking the manager), but we can
monitor this in the future and adjust as needed if manager-calculations are
expensive. 

We outline the behavior of the manager and worker within \libE:

\noindent \textbf{Manager}
\begin{itemize}
  \item Generates inputs for calculations.
    \begin{itemize}
      \item If the calculation is a simulation, determine parameters to be
        evaluated.
      \item If the calculation is a local optimization run, give state
        information for determining the next point in a run. 
      \item If the calculation is deciding where to start a run, then give all
        evaluated (and about-to-be-evaluated) points.
    \end{itemize}
  \item Launches calculations
    \begin{itemize}
      \item Possibly resume previously paused calculations
    \end{itemize}
  \item Received output from calculations
    \begin{itemize}
      \item Possibly receive intermediate output
    \end{itemize}
  \item Coordinates concurrent calculations 
    \begin{itemize}
      \item Dynamic queue of pending calculations
      \item Possibly monitor sims
    \end{itemize}
  \item Tracks history of calculations
  \item Allocated resource to calculations 
    \begin{itemize}
      \item Possibly changes resources during calculations (or just simulations)
    \end{itemize}
  \item Tolerates failed calculations (or just simulations)
    \begin{itemize}
      \item Possibly without losing the resource when possible
    \end{itemize}
  \item Can exploit persistent data
    \begin{itemize}
      \item For example: checkpointing, meshes, iterative solvers
    \end{itemize}
  \item Can gracefully terminate calculations
    \begin{itemize}
      \item Possibly pause
    \end{itemize}
\end{itemize}


\noindent \textbf{Worker}
\begin{itemize}
  \item Performs the calculations given to it.
  \item An undivisible unit (though many workers may combine to perform one task). Examples:
    \begin{itemize}
      \item If the simulation is an MPI executable, one worker may call 
        \begin{center}
          \texttt{mpiexec -np 32 -machinefile two\_nodes a.out}
        \end{center}
      \item If the simulation accepts an MPI subcommunicator, many workers may
        form a subcommunicator, and all are involved in the simulation
        evaluation (and one worker will report back to the manager).
    \end{itemize}
\end{itemize}


\section{Pseudocode}

We outline the logic of the the manager and the workers in \algref{manager} and \algref{worker}, respectively.

\LinesNumbered
\begin{algorithm}[H] 
  \SetKwComment{Comment}{$\triangleright$\ }{}
  \SetAlgoNlRelativeSize{-5}
  \SetKwInOut{Input}{input}
  \SetKwInOut{Optional}{optional}
  \SetKwInOut{Output}{output}
  \SetKw{true}{true}
  \SetKw{false}{false}
  \Input{Simulations $\mathtt{sim}_1,\ldots,\mathtt{sim}_{n_s}$, \newline
         Point generating function $\mathtt{gen}$, \newline
         $c$ workers
  }
  \Optional{Termination test $\mathtt{term\_test}$ \newline
            Active simulation and queue update function $\mathtt{update\_active\_and\_queue}$ \newline
            Function to decide calculations and resources $\mathtt{decide\_work\_and\_resources}$
  }
  Initialize history $H$ (information about past and current and possible future calculations)\\

  \While{$\mathtt{term\_test}(H)$}
  {
    \While{Any worker is waiting to return $\mathtt{sim}$ or $\mathtt{gen}$ results} 
    {
      Receive from all workers with $\mathtt{sim}$ and $\mathtt{gen}$ work\\
    }

    $(H) = \mathtt{update\_active\_and\_queue}(H)$\Comment*[r]{blocks manager}
    $(W_s, W_g) = \mathtt{decide\_work\_and\_resources}(H)$\Comment*[r]{blocks manager}

    \For{Each idle worker $i$}
    {
      \eIf{ worker $i \in W_s$}
      {
      Tell worker $i$ to evaluate $\mathtt{sim}_j$ at point(s) from $H$ with appropriate resources
      }{
      Begin worker $i$ on $\mathtt{gen}(H)$\\
      }
    }
  }
  Receive from active workers\\ 
  Terminate all workers
  \caption{\libE manager logic \label{alg:manager}} 
\end{algorithm}

\LinesNumbered
\begin{algorithm}[H]
  \SetKwComment{Comment}{$\triangleright$\ }{}
  \SetAlgoNlRelativeSize{-5}
  \SetKwInOut{Input}{input}
  \SetKw{true}{true}
  \SetKw{false}{false}
  \SetKw{none}{none}
  \Input{\none}

  \SetKw{break}{break}
  \While{\true}
  {
    $D =$ Receive from manager \\ 
    \If{$D.tag == \mathtt{stop\_tag}$ }{\break}

    \If{$D.form\_subcomm$ is nonempty}{Form subcommunicator $sc$ with other workers}

    \If{Necessary parts of calculation are not initialized}{(Collectively) initialize}

    $O = D.calc\_f(sc, D.calc\_params)$

    \If{First element in subcommunicator $sc$}{Report $O$ to manager}
    
  }
  \caption{Each \libE worker's logic \label{alg:worker}} 
\end{algorithm}

\section{API}
We now outline the API for \libE and the two functions expected by \libE: the
simulation and the generating functions.

\begin{allintypewriter}
  libE(c, allocation\_specs, sim\_specs, failure\_processing, exit\_criteria)\\
  % history,

  \begin{description}
    \item[c]: [dict] 
      \begin{description}
        \item['comm']: [mpi4py communicator] to be used by libE
        \item['color']: [int] communicator color
      \end{description}

    \item[allocation\_specs]: [dict]
      \begin{description}
        \item[manager\_ranks]: [set of ints] 
        \item[worker\_ranks]: [set of ints]
        \item[machinefile]:
      \end{description}

    \item[sim\_specs]: [dict] 
      \begin{description}
        \item[f]: [list of funcs] that calls each sim(s)
        \item[in]: [list] string keys that sim wants from history (assumed common to all sims)
        \item[out]: [list of tuples] (string keys, type, [size>1]) sim outputs (assumed common to all sims)
        \item[params]: [list of dicts] parameters for each f
      \end{description}

    \item[gen\_specs]: [dict] 
      \begin{description}
        \item[f]: [func] generates points to be evaluated by a sim
        \item[in]: [list] string keys that gen wants from history
        \item[out]: [list of tuples] (string keys, type, [size>1]) of gen outputs
        \item[params]: [dict] additional parameters for gen\_f. 
          % E.g.: 
          % \begin{itemize}
          %   \item[lb]: [n-by-1 array] lower bound on sim parameters
          %   \item[ub]: [n-by-1 array] upper bound on sim parameters
          % \end{itemize}
      \end{description}

    \item[failure\_processing]: [dict]

    \item[exit\_criteria]: [dict] with possible fields:
      \begin{description}
        \item[sim\_eval\_max]: [int] Stop after this many evaluations.
        \item[min\_sim\_f\_val]: [dbl] Stop when a value below this has been found.
      \end{description}

    % \item[history]: [numpy structured array] 
    %   \begin{description}
    %     \item[x]: parameters given to simulation(s)
    %     \item[f]: simulation value(s) at each x
    %     \item[...]:
    %     \item[...]:
    %   \end{description}

  \end{description}
\end{allintypewriter}
        

\subsection{\texttt{sim} API}
The \texttt{sim} calculations will be called by \libE with the following API:\\

\begin{allintypewriter}
  out = sim\_f(H[sim\_specs['in']][max\_priority\_inds], sim\_specs['out'],
  params)\\
\end{allintypewriter}

where \texttt{out} is a numpy structured array with keys/value-sizes matching
those in \texttt{sim\_specs[out]}. Note that if \texttt{gen\_f} produces points
with equal priorities, they will given in a batch to a worker.

\subsection{\texttt{gen} API}
The \texttt{gen} calculations will be called by \libE with the following API:\\

\begin{allintypewriter}
  out = gen\_f(H[gen\_spec['in']][:H\_ind], gen\_specs['out'], params)\\
\end{allintypewriter}

One of the fields in \texttt{out} must be \texttt{'priority'}. Note that all
points with the same priority are given to \texttt{sim\_f} at a time.

Note that \texttt{sim\_f} receives only a subset of generated points, but
\texttt{gen\_f} receives information from all points (if \texttt{gen\_f['in']}
is nonempty).

\subsection{Notes:}
The following strings cannot be used in  \texttt{gen\_specs['out']} or
\texttt{sim\_specs['out']} because they are reserved for \libE use:
\begin{itemize}
  \item \texttt{'priority'}
  \item \texttt{'given'}
  \item \texttt{'given\_time'}
  \item \texttt{'lead\_rank'}
  \item \texttt{'returned'}
  \item \texttt{'sim\_id'} : The generator can assign this, but users should be
    careful doing so.
\end{itemize}

All values in \texttt{gen\_specs['out']} or \texttt{sim\_specs['out']} are
initialized to 0 (or FALSE for booleans). Just know this, for example, when
assigning 'run\_id' within scripts like \texttt{gen\_specs['f']}.

Internally, \libE currenlty maintains a single data structure \texttt{H} which
contains the all history information (from \texttt{sim\_specs['out'] +
gen\_specs['out']}). 

We have considered splitting the history \texttt{H} into multiple data structures. One possible split:

\begin{allintypewriter}
  \begin{itemize}
    \item[H\_in]: [numpy structured array] History of all input given to
      sim\_f. Rows correspond to each ``simulation evaluation''. Contains fields
      in sim\_specs['in']. 
      
    \item[H\_out]: [numpy structured array] History of all simulation output
      and derived quantities. Contains fields in sim\_specs['out']+gen\_specs['out'] \ sim\_specs['in']
  \end{itemize}
\end{allintypewriter}

Reasons for making some type of split from \texttt{H} to \texttt{H\_in} and \texttt{H\_out}:
\begin{description}
  \item[When there are repeated evaluations:]\
    \begin{itemize}
      \item Deterministic simulations:
        \begin{itemize}
          \item[+] It is possible to remove a redudant entries from \texttt{H}. E.g.,
            \begin{allintypewriter}
              \begin{itemize}
                \item H['x']
                \item H['f']
                \item H['dist\_to\_unit\_bounds']
                \item H['ind\_of\_better\_s']
                \item H['ind\_of\_better\_l']
                \item H['dist\_to\_better\_s']
                \item H['dist\_to\_better\_l']
              \end{itemize}
            \end{allintypewriter}
          \item[-] Who decides what is put in different ``history parts''?
        \end{itemize}
      \item Stochastic simulations:
        \begin{itemize}
          \item[+] It is possible to remove redudant entries from \texttt{H}
            (though fewer than in the deterministic case).
          \item[-] Now \texttt{H\_in} entries will need to be duplicated and
            cross referencing output in \texttt{H\_out} becomes harder. E.g.,
            if \texttt{'replications'} is an entry in \texttt{sim\_specs['in']}
            then there will still be many rows in \texttt{H\_out}.
        \end{itemize}
    \end{itemize}
\end{description}

At this time, it seems as if strongest case for spliting is if the simulation
input is very large and stored within \texttt{H} instead of pointers to
existing structures/files/meshes. And even in the case of stochastic
simulations with replications, calling \texttt{sim\_f(x=$x_0$, reps=4)} and
\texttt{sim\_f(x=$x_0$, reps=3)} will require two entries in \texttt{H\_in} and
7 entries in \texttt{H\_out}.

We'll keep everything in \texttt{H} for now, and consider splitting if we
encounter many such use cases.

\clearpage
\subsection{Example of calling functions}
\lstinputlisting[basicstyle=\footnotesize,frame=single,language=Python,title=\lstname,numberstyle=\tiny,numbers=left,firstline=43]{../../code/examples/GKLS_and_uniform_random_sample/call_libE_on_GKLS.py}

\clearpage
\lstinputlisting[basicstyle=\footnotesize,frame=single,language=Python,title=\lstname,numberstyle=\tiny,numbers=left,firstline=23]{../../code/examples/chwirut_and_aposmm/call_libE_on_chwirut.py}

\clearpage
\section{Target problems}
There are many types of simulations that we can consider being run in \libE.

\begin{enumerate}
  \item A Python function
    \begin{itemize}
      \item We assume this is thread-safe.
      \item Use cases: 
        \begin{itemize}
          \item GKLS problem generator
          \item chwirut1.py
        \end{itemize}
    \end{itemize}
  \item An executable
    \begin{itemize}
      \item May use MPI
      \item Must perform evaluations in a manner that won't conflict with other evaluations. 
        \begin{itemize}
          \item Performs read/writes in the directory where it is run (or in a given directory)
        \end{itemize}
      \item Must be able to tell executable which resources to use.
        \begin{itemize}
          \item MCS compute node: Can specify CPU
          \item Blues: Can specify machinefile
          \item Mira: Not going to be addressed by \libE.
          \item Theta/Aurora: Unsure how to accomplish this at this time.
          \item Cray system: Unsure how to accomplish this at this time.
        \end{itemize}
      \item Use case: 
        \begin{itemize}
          \item OPAL accelerator simulation [John Power and Nicole Neveu]
          \item LAMMPS simulation [Simon Phillpot and Eugene Ragasa]
        \end{itemize}
    \end{itemize}
  \item An MPI simulation with a subcommunicator
    \begin{itemize}
      \item Possibly stops regularly to communicate with manager
      \item Use cases: 
        \begin{itemize}
          \item Possibly the HFBTHO simulation 
        \end{itemize}
    \end{itemize}
  \item PETSc simulation 
    \begin{itemize}
      \item Access to complete memory stack
      \item Easier to kill/monitor?
      \item Use cases: 
        \begin{itemize}
          \item Still considering different possibilities.
        \end{itemize}
    \end{itemize}
\end{enumerate}


\clearpage
\section{Initial test cases}
In order to guide the initial development of \libE, we will focus on supporting the following use cases. (Objectives are intentionally selected to be easy to evaluate.)
\begin{enumerate}
  \item 
    \begin{description}
      \item[Objective:] GKLS
      \item[Generating function:] Uniform sampling on $[0,1]^n$ with different batch sizes
      \item[Functionality tested:] \
        \begin{itemize}
          \item Handling of different numbers of points given/returned from workers
        \end{itemize}
      \item[Status:] Completed
    \end{description}
    \bigskip
  \item 
    \begin{description}
      \item[Objective:] chwirut1.py
      \item[Generating function:] Multiple POUNDERS runs from $K$ starting points.
      \item[Functionality tested:] \
        \begin{itemize}
          \item APOSMM in \libE
        \end{itemize}
      \item[Status:] Completed
    \end{description}
    \bigskip
  \item 
    \begin{description}
      \item[Objective:] chwirut1.py
      \item[Generating function:] Same as above, but with APOSMM giving each point and a single residual to be evaluated.
      \item[Functionality tested:] \
        \begin{itemize}
          \item Being able to give different residuals in APOSMM 
        \end{itemize}
      \item[Status:] Completed
    \end{description}
    \bigskip
  \item 
    \begin{description}
      \item[Objective:] HFBTHO (imbalance)/variable internal tols
      \item[Generating function:] POUNDERS with adaptive tolerance attempts
      \item[Functionality tested:] \
        \begin{itemize}
          \item Variable time to evaluate each residual
        \end{itemize}
      \item[Status:] Currently working with Jason Sarich to get this implemented.
    \end{description}
    \bigskip
  \item 
    \begin{description}
      \item[Objective:] Eldad and Lauren subsurface (in TAO) 
      \item[Generating function:] LCAL PDECO Stefan 
      \item[Functionality tested:] \
        \begin{itemize}
          \item Stefan ??
        \end{itemize}
      \item[Status:] Looking for a sample average approximation method that can be used to generate points to be evaluated.
    \end{description}
    \bigskip
  \item 
    \begin{description}
      \item[Objective:] chwirut1.py with stochastic noise on each residual
      \item[Generating function:] POUNDERS using sample mean with the number of replications determined by iteration number. 
      \item[Functionality tested:] \
        \begin{itemize}
          \item Efficient handling of multiple evaluations of single points
        \end{itemize}
      \item[Status:] Looking for a sample average approximation method that can be used to generate points to be evaluated.
    \end{description}
\end{enumerate}

\cite{nlopt}

\bibliographystyle{plain}
\bibliography{../bibs/masterbib}
\end{document} 
