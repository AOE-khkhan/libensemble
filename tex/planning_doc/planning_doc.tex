\documentclass{article}

\usepackage{listings}
\usepackage{amsmath, amsfonts, amssymb, setspace, color, graphicx}%psfrag}%,pstool}
\usepackage[american]{babel}
\usepackage[normalem]{ulem} % For \sout
\usepackage{fullpage}
\everymath{\displaystyle}
\usepackage[ruled,vlined]{algorithm2e}

\newenvironment{allintypewriter}{\ttfamily}{\par}
\newcommand{\jlnote}[1]{\textsf{{\color{blue}{ JL note:}   #1.} }\marginpar{{\textbf{Comment}}}}
\newcommand{\ps}[1]{\ensuremath{\mathtt{#1}}}

\title{LibEnsemble planning document}

\begin{document}
\maketitle
\begin{abstract}
  This document will be an explicit plan of what will be developed in
  LibEnsemble. Proper planning should help ensure that the code will be flexible
  and easy to adjust in the future.

  Current development considers a manager and worker framework. 
\end{abstract}

In order to ensure applicability to a variety of use cases, 
LibEnsemble will coordinate many different types of calculations. Examples of
such calculations include:
\begin{itemize}
  \item Simulation evaluations
  \item Generation of point(s) to be evaluated by the simulation(s)
  \item Updating points in the active queue
  \item Updating active simulations
  \item Monitoring intermediate output from simulations
\end{itemize}
Users will only need to provide scripts for simulation evaluation and point
generation. We will provide default scripts for the other types of calculations, 
which can be templates for the interested user. As a default the Manager will
perform many calculations itself (essentially blocking the manager), but we can
possibly monitor this and adjust as needed if manager-calculations are
expensive. 

\subsection{Manager}
\begin{itemize}
  \item Generates inputs for calculations.
    \begin{itemize}
      \item If the calculation is a simulation, determine parameters to be
        evaluated.
      \item If the calculation is a local optimization run, give state
        information for determining the next point in a run. 
      \item If the calculation is deciding where to start a run, then give all
        evaluated (and about-to-be-evaluated) points.
    \end{itemize}
  \item Launches calculations
    \begin{itemize}
      \item Possibly resume previously paused calculations
    \end{itemize}
  \item Received output from calculations
    \begin{itemize}
      \item Possibly receive intermediate output
    \end{itemize}
  \item Coordinates concurrent calculations 
    \begin{itemize}
      \item Dynamic queue of pending calculations
      \item Possibly monitor sims
    \end{itemize}
  \item Tracks history of calculations
  \item Allocated resource to calculations 
    \begin{itemize}
      \item Possibly changes resources during calculations (or just simulations)
    \end{itemize}
  \item Tolerates failed calculations (or just simulations)
    \begin{itemize}
      \item Possibly without losing the resource when possible
    \end{itemize}
  \item Can exploit persistent data
    \begin{itemize}
      \item For example: checkpointing, meshes, iterative solvers
    \end{itemize}
  \item Can gracefully terminate calculations
    \begin{itemize}
      \item Possibly pause
    \end{itemize}
\end{itemize}

If we are going to keep everything general, then perhaps we require a
calculation-type-dependent user-defined function to be called before and after
after every type of calculation.

\subsection{Worker}
\begin{itemize}
  \item Performs the calculations given to it.
  \item An undivisible unit (though many workers may combine to perform one task). Examples:
    \begin{itemize}
      \item If the simulation is an MPI executable, one worker may call 
        \begin{center}
          \texttt{mpiexec -np 32 -machinefile two\_nodes a.out}
        \end{center}
      \item If the simulation accepts an MPI subcommunicator, many workers may
        form a subcommunicator, and all are involved in the simulation
        evaluation (and one worker will report back to the manager).
    \end{itemize}
\end{itemize}


\section{Pseudocode}

\newpage
\LinesNumbered
\begin{algorithm}[t] % Makes font smaller
% \begin{algorithm}[H] % Makes font larger
  \SetKwComment{Comment}{$\triangleright$\ }{}
  \SetAlgoNlRelativeSize{-5}
  \SetKwInOut{Input}{input}
  \SetKwInOut{Optional}{optional}
  \SetKwInOut{Output}{output}
  \SetKw{true}{true}
  \SetKw{false}{false}
  \Input{Simulations $\mathtt{sim}_1,\ldots,\mathtt{sim}_{n_s}$, \newline
         Point generating function $\mathtt{gen}$, \newline
         $c$ workers
  }
  \Optional{Termination test $\mathtt{term\_test}$ \newline
            Active simulation and queue update function $\mathtt{update\_active\_and\_queue}$ \newline
            Function to decide calculations and resources $\mathtt{decide\_work\_and\_resources}$
  }
  Initialize history $H$ (past and current calculations)\\
  Initialize queue $Q$ (values and resources for to-be-evaluated points) 

  \While{$\mathtt{term\_test}(H,Q)$}
  {
    \While{Any worker is waiting to return $\mathtt{sim}$ or $\mathtt{gen}$ results} 
    {
      Receive from all workers with $\mathtt{sim}$ and $\mathtt{gen}$ work\\
    }

    $(H,Q) = \mathtt{update\_active\_and\_queue}(H,Q)$\Comment*[r]{blocks manager}
    $(W_s, W_g, Q) = \mathtt{decide\_work\_and\_resources}(H,Q)$\Comment*[r]{blocks manager}

    \For{Each idle worker $i$}
    {
      \eIf{ worker $i \in W_s$}
      {
      Tell worker $i$ to evaluate $\mathtt{sim}_j$ at point(s) from $Q$ with appropriate resources
      }{
      Begin worker $i$ on $\mathtt{gen}(H,Q)$\\
      }
    }
  }
  Receive from (kill?) active workers\\ 
  Terminate all workers
  \caption{LibEnsemble manager logic \label{alg:manager}} 
\end{algorithm}

\newpage
\LinesNumbered
\begin{algorithm}[t] % Makes font smaller
% \begin{algorithm}[H] % Makes font larger
  \SetKwComment{Comment}{$\triangleright$\ }{}
  \SetAlgoNlRelativeSize{-5}
  \SetKwInOut{Input}{input}
  \SetKw{true}{true}
  \SetKw{false}{false}
  \SetKw{none}{none}
  \Input{\none}

  \SetKw{break}{break}
  \While{\true}
  {
    $D =$ Receive from manager \\ 
    \If{$D.tag == \mathtt{stop\_tag}$ }{\break}

    \If{$D.form\_subcomm$ is nonempty}{Form subcommunicator $sc$ with other workers}

    \If{Necessary parts of calculation are not initialized}{(Collectively) initialize}

    $O = D.calc\_f(sc, D.calc\_params)$

    \If{First element in subcommunicator $sc$}{Report $O$ to manager}
    
  }
  \caption{Each LibEnsemble worker's logic \label{alg:manager}} 
\end{algorithm}

\section{API}

\subsection{\texttt{LibEnsemble} API}

\begin{allintypewriter}
  libE(c, allocation\_specs, sim\_specs, failure\_processing, exit\_criteria)\\
  % history,

  \begin{description}
    \item[c]: [dict] 
      \begin{description}
        \item['comm']: [mpi4py communicator] to be used by libE
        \item['color']: [int] communicator color
      \end{description}

    \item[allocation\_specs]: [dict]
      \begin{description}
        \item[manager\_ranks]: [set of ints] 
        \item[worker\_ranks]: [set of ints]
        \item[machinefile]:
      \end{description}

    \item[sim\_specs]: [dict] 
      \begin{description}
        \item[f]: [list of funcs] that calls each sim(s)
        \item[in]: [list] string keys that sim wants from history (assumed common to all sims)
        \item[out]: [list of tuples] (string keys, type, [size>1]) sim outputs (assumed common to all sims)
        \item[params]: [list of dicts] parameters for each f
      \end{description}

    \item[gen\_specs]: [dict] 
      \begin{description}
        \item[f]: [func] generates points to be evaluated by a sim
        \item[in]: [list] string keys that gen wants from history
        \item[out]: [list of tuples] (string keys, type, [size>1]) of gen outputs
        \item[params]: [dict] additional parameters for gen\_f. 
          % E.g.: 
          % \begin{itemize}
          %   \item[lb]: [n-by-1 array] lower bound on sim parameters
          %   \item[ub]: [n-by-1 array] upper bound on sim parameters
          % \end{itemize}
      \end{description}

    \item[failure\_processing]: [dict]

    \item[exit\_criteria]: [dict] with possible fields:
      \begin{description}
        \item[sim\_eval\_max]: [int] Stop after this many evaluations.
        \item[min\_sim\_f\_val]: [dbl] Stop when a value below this has been found.
      \end{description}

    % \item[history]: [numpy structured array] 
    %   \begin{description}
    %     \item[x]: parameters given to simulation(s)
    %     \item[f]: simulation value(s) at each x
    %     \item[...]:
    %     \item[...]:
    %   \end{description}

  \end{description}
\end{allintypewriter}
        

\subsection{\texttt{sim} API}
The \texttt{sim} calculations must have the API:\\

\begin{allintypewriter}
  out = sim\_f(H[sim\_specs['in']][max\_priority\_inds], sim\_specs['out'],
  params)\\
\end{allintypewriter}

where \texttt{out} is a numpy structured array with keys/value-sizes matching
those in \texttt{sim\_specs[out]}. Note that if \texttt{gen\_f} produces points
with equal priorities, they will given in a batch to a worker.

\subsection{\texttt{gen} API}
The \texttt{gen} calculations must have the API:\\

\begin{allintypewriter}
  out = gen\_f(H[gen\_spec['in']][:H\_ind], gen\_specs['out'], params)\\
\end{allintypewriter}

Note that \texttt{sim\_f} receives only a subset of generated points, but
\texttt{gen\_f} receives information from all points (if \texttt{gen\_f['in']}
is nonempty).



\clearpage
\subsection{Example of calling functions}
\lstinputlisting[basicstyle=\footnotesize,frame=single,language=Python,title=\lstname,numberstyle=\tiny,numbers=left,firstline=43]{../../examples/GKLS_and_uniform_random_sample/call_libE_on_GKLS.py}

\clearpage
\lstinputlisting[basicstyle=\footnotesize,frame=single,language=Python,title=\lstname,numberstyle=\tiny,numbers=left,firstline=23]{../../examples/chwirut_and_multiple_pounders/call_libE_on_chwirut.py}

\clearpage
\section{Target problems}
There are many types of simulations that we can consider being run in LibEnsemble.

\begin{enumerate}
  \item A Python function
    \begin{itemize}
      \item We assume this is thread-safe.
      \item Use cases: 
        \begin{itemize}
          \item ??
        \end{itemize}
    \end{itemize}
  \item An executable
    \begin{itemize}
      \item May use MPI
      \item Must perform evaluations in a manner that won't conflict with other evaluations. 
        \begin{itemize}
          \item Performs read/writes in the directory where it is run (or in a given directory)
        \end{itemize}
      \item Must be able to tell executable where to run.
        \begin{itemize}
          \item MCS compute node: Can specify CPU
          \item Blues: Can specify machinefile
          \item Mira: Not going to be addressed at this time.
          \item Theta/Aurora: ?
          \item Cray system: ?
        \end{itemize}
      \item Use case: 
        \begin{itemize}
          \item OPAL accelerator simulation [John Power and Nicole Neveu]
          \item LAMMPS simulation [Simon Phillpot and Eugene Ragasa]
        \end{itemize}
    \end{itemize}
  \item An MPI simulation with a subcommunicator
    \begin{itemize}
      \item Possibly stops regularly to communicate with manager
      \item Use cases: 
        \begin{itemize}
          \item ??
        \end{itemize}
    \end{itemize}
  \item PETSc simulation 
    \begin{itemize}
      \item Access to complete memory stack
      \item Easier to kill/monitor?
      \item Use cases: 
        \begin{itemize}
          \item ??
        \end{itemize}
    \end{itemize}
\end{enumerate}


\clearpage
\section{Initial test cases}
In order to guide the initial development of LibEnsemble, we will focus on supporting the following use cases. (Objectives are intentionally selected to be easy to evaluate.)
\begin{enumerate}
  \item 
    \begin{description}
      \item[Objective:] GKLS
      \item[Generating:] Uniform sampling on $[0,1]^n$ with different batch sizes
      \item[Functionality:] \
        \begin{itemize}
          \item Handling of different numbers of points given/returned from workers
        \end{itemize}
    \end{description}
    \bigskip
  \item 
    \begin{description}
      \item[Objective:] chwirut\{1,2\}.c 
      \item[Generating:] Single TAO/POUNDERS run from 1 starting point. One simulation for each residual. Residuals obtained concurrently.
      \item[Functionality:] \
        \begin{itemize}
          \item Multiple simulations for each point
        \end{itemize}
    \end{description}
    \bigskip
  \item 
    \begin{description}
      \item[Objective:] chwirut\{1,2\}.c
      \item[Generating:] Multiple POUNDERS runs from $K$ starting points.
      \item[Functionality:] \
        \begin{itemize}
          \item APOSMM in LibEnsemble
        \end{itemize}
    \end{description}
    \bigskip
  \item 
    \begin{description}
      \item[Objective:] HFBTHO (imbalance)/variable internal tols
      \item[Generating:] POUNDERS with adaptive tolerance attempts
      \item[Functionality:] \
        \begin{itemize}
          \item Stefan ??
        \end{itemize}
    \end{description}
    \bigskip
  \item 
    \begin{description}
      \item[Objective:] Eldad and Lauren subsurface (in TAO) 
      \item[Generating:] LCAL PDECO Stefan ??
      \item[Functionality:] \
        \begin{itemize}
          \item Stefan ??
        \end{itemize}
    \end{description}
    \bigskip
  \item 
    \begin{description}
      \item[Objective:] chwirut\{1,2\}.c with stochastic noise on each residual
      \item[Generating:] POUNDERS using sample mean with the number of replications determined by iteration number. 
      \item[Functionality:] \
        \begin{itemize}
          \item Efficient handling of multiple evaluations of single points
        \end{itemize}
    \end{description}
\end{enumerate}


% \bibliographystyle{plain}
% \bibliography{article}
\end{document} 
